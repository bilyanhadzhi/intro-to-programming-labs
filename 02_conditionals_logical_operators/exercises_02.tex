\documentclass[12pt]{article}
\usepackage[utf8]{inputenc}
\usepackage[T1,T2A]{fontenc}
\usepackage[bulgarian]{babel}
\usepackage{inconsolata}
\usepackage{hyperref}

\hypersetup{
  colorlinks=true,
  linkcolor=blue,
  urlcolor=blue
}

\newcommand{\example}[2] {

    \smallskip
    \underline{Пример}:

    Вход:\hphantom{д} \code{#1}

    Изход: \code{#2}
}

\newcommand{\explanation}[1] {

    \underline{Пояснение}: #1
}

\newcommand{\exercise}[2] {
    \textbf{Зад. #1.} #2
}

\newcommand{\latinttfamily}{\fontencoding{OT1}\ttfamily}
\DeclareTextFontCommand{\ltexttt}{\latinttfamily}
\newcommand{\code}[1]{\ltexttt{#1}}


\title{Увод в програмирането 2020/2021\\ Група 8, практикум №2}
\date{}

\begin{document}
\maketitle

\exercise{1}{
    Да се напише програма, която приема две цели числа \code{x} и \code{y} и извежда
    на конзолата стойността на |\code{x} - \code{y}|.
}

\bigskip
\exercise{2}{
    При вход три цели числа от конзолата, да се изведе най-голямото от тях.
}

\bigskip
\exercise{3}{
    При вход две числа от тип \code{double}, които ще считаме за координати на точка в равнината,
    да се изведе в кой квадрант се намира точката.
}

\example{1 -4}{IV}

\bigskip
\exercise{4}{
    При вход цяло положително число, което ще приемем за година, да се изпише
    на екрана дали съответната година е високосна.
}
\explanation{
    Една година е високосна, ако е вярно някое от следните:
    \begin{itemize}
        \item дели се на 4, но не се дели на 100;
        \item дели се на 400.
    \end{itemize}

}

\bigskip
\exercise{5}{
    При вход - цяло число в интервала [1, 7], да се изведе на кой ден от седмицата съответства то.
}
\example{5}{Friday}
\example{9}{Noday}

\newpage
\exercise{6}{
    Да се напише програма, която приема 4 цели числа, считайки първите две за ден и месец на дата,
    а последните две - ден и месец на друга дата. Да се изведе на стандартния изход дали първата
    дата е преди втората.
}

\example{15 9 25 12}{1}
\explanation{В случая първата дата е 15.09, а втората – 25.12.}
\example{1 3 10 2}{0}

\bigskip
\exercise{7}{
    Да се напише програма, която при вход цяло число в интервала [0, 100], изписва
    на каква оценка съответства конкретният брой точки.

    Ръководете се от следната таблица:

    \begin{tabular}{ |c|c|c| }
        \hline
        Точки & Оценка \\
        \hline
        < 0 & \code{Impossible!} \\
        \hline
        [0, 59] & \code{2} \\
        \hline
        [60, 69] & \code{3} \\
        \hline
        [70, 79] & \code{4} \\
        \hline
        [80, 89] & \code{5} \\
        \hline
        [90, 100] & \code{6} \\
        \hline
        > 100 & \code{Incredible!} \\
        \hline
    \end{tabular}
}
\example{75}{4}

\bigskip
\exercise{8}{
    Да се напише програма, която приема три цели страни на триъгиълник.
    \begin{itemize}
        \item Ако страните не образуват триъгълник, да се изпише \code{Invalid sides};
        \item В противен случай, да се изпише:
        \begin{itemize}
            \item \code{equilateral}, ако триъгълникът е равностранен;
            \item \code{isosceles}, ако е равнобедрен;
            \item \code{scalene}, ако е разностранен.
        \end{itemize}
    \end{itemize}
}

\newpage
\exercise{9}{
    По дадено цяло трицифрено число \code{n}, да се изпише дали е вярно, че \code{n} се дели на всичките си цифри.

    *Предполагаме, че \code{n} не съдържа 0 като цифра.
}
\example{128}{1}
\explanation{\code{128 \% 1 == 0}, \code{128 \% 2 == 0}, \code{128 \% 8 == 0}}

\bigskip
\exercise{10}{
    Да се напише програма, която приема 6 цели числа, съответстващи на координатите на три точки.
    Да се изпише на конзолата дали тези три точки лежат на една права.
}

\end{document}
