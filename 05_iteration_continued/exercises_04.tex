\documentclass[12pt]{article}
\usepackage[utf8]{inputenc}
\usepackage[T1,T2A]{fontenc}
\usepackage[bulgarian]{babel}
\usepackage{inconsolata}
\usepackage{hyperref}
\usepackage{indentfirst}

\hypersetup{
  colorlinks=true,
  linkcolor=blue,
  urlcolor=blue
}

\newcommand{\example}[2] {

    \smallskip
    \underline{Пример}:

    Вход:\hphantom{д} \code{#1}

    Изход: \code{#2}
}

\newcommand{\explanation}[1] {

    \underline{Пояснение}: #1
}

\newcommand{\outputline}[1] {\\\indent \indent #1}

\newcommand{\exercise}[2] {
    \textbf{Зад. #1.} #2
}

\newcommand{\latinttfamily}{\fontencoding{OT1}\ttfamily}
\DeclareTextFontCommand{\ltexttt}{\latinttfamily}
\newcommand{\code}[1]{\ltexttt{#1}}


\title{Увод в програмирането 2020/2021\\ Група 8, практикум №4}
\date{}

\begin{document}
\maketitle

\exercise{1}{
    Да се напише програма, която приема число и обръща знака му чрез побитови операции.
}
\bigskip

\exercise{2}{
    Да се напише програма, която по подадено неотрицателно число пресмята за колко стъпки
    може да се превърне в 0. Правилата са следните:
    \begin{itemize}
        \item Ако числото е четно, го делим на 2.
        \item Ако числото е нечетно, изваждаме 1 от него.
    \end{itemize}
}

\exercise{3}{
    Да се напише програма, която приема естествено число \code{n} и \code{n} на брой
    числа. Като резултат да се изведе сумата от квадратите на всички числа.
}
\example{\outputline{3} \outputline{3 -1 4}}{26}
\bigskip

\exercise{4}{
    Да се напише програма, която приема естествено число \code{n}
    и извежда разликата:
    $$(1 + 2 + \ldots + n)^2 - (1^2 + 2^2 + \ldots + n^2)$$
}
\example{5}{170}
\explanation{
    \outputline{$(1 + 2 + \ldots + 5)^2 = 15^2 = 225$}
    \outputline{$1^2 + 2^2 + \ldots + 5^2 = 55$}
}
\bigskip

\exercise{5*}{
    Да се напише програма, която по въведено число проверява дали може да се
    получи като сума на две прости числа и извежда всички варианти на конзолата.
}
\bigskip

\exercise{6**}{
    Дадено е цяло неотрицателно число и възможността да се разменят само две от цифрите му.
    Да се намери най-голямото число, което може да се образува.
}
\example{4753}{7453}
\example{7266}{7662}
\example{854}{854}

\end{document}
