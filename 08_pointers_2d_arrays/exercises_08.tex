\documentclass[12pt]{article}
\usepackage[utf8]{inputenc}
\usepackage[T1,T2A]{fontenc}
\usepackage[bulgarian]{babel}
\usepackage{inconsolata}
\usepackage{hyperref}
\usepackage{indentfirst}
\usepackage{multicol}

\hypersetup{
  colorlinks=true,
  linkcolor=blue,
  urlcolor=blue
}

\newcommand{\example}[2] {

    \smallskip
    \underline{Пример}:

    Вход:\hphantom{д} \code{#1}

    Изход: \code{#2}
}

\newcommand{\explanation}[1] {

    \underline{Пояснение}: #1
}

\newcommand{\outputline}[1] {\\\indent \indent #1}

\newcommand{\exercise}[2] {
    \textbf{Зад. #1.} #2
}

\newcommand{\latinttfamily}{\fontencoding{OT1}\ttfamily}
\DeclareTextFontCommand{\ltexttt}{\latinttfamily}
\newcommand{\code}[1]{\ltexttt{#1}}


\title{Увод в програмирането 2020/2021\\ Група 8, практикум №8}
\date{}

\begin{document}
\maketitle

\exercise{1}{Да се напише функция, които извежда стойностите на масив с дължина \code{n}.
Използвайте \textbf{указателна аритметика}.}
\bigskip

\exercise{2}{Да се напише функция, която разменя стойностите на две променливи.}
\bigskip

\exercise{3}{Да се напише функция, която пренарежда елементите на масив така, че на първо място
да стои най-големият, след него – най-малкият, след това – вторият най-голям и т.н.}
\example{[5, 2, 1, 6, 7, 4]}{[7, 1, 6, 2, 5, 4]}
\bigskip

\exercise{4}{Да се напише функция \code{add}, която приема три квадратни матрици и записва
сбора на първата и втората в третата.}
\bigskip

\exercise{5}{Да се напише функция, която при подаден масив извежда на екрана всеки елемент,
за който е изпълнено, че:
    \begin{itemize}
        \item е най-малък в реда си;
        \item е най-голям в колоната си.
    \end{itemize}
}
\bigskip

\exercise{6}{Да се напише функция, която при подадена квадратна матрица извежда дали
тя е магически квадрат.
}
\bigskip

\newpage

\exercise{7}{Дадена е матрица от $n \times n$ цели числа.
Да се разменят всички колони с четни индекси (като броенето започва от 0) със съответните им редове.
Разменянето става последователно. Да се изведе резултатът.
}
\example{
    \outputline{1 2 3}
    \outputline{4 5 6}
    \outputline{7 8 9}
}{
    \outputline{1 4 3}
    \outputline{2 5 8}
    \outputline{7 6 9}
}
\bigskip

\exercise{8}{(Домашна работа)

Получавате част от кода за играта \href{https://en.wikipedia.org/wiki/15_puzzle}{\textbf{Пъзел 15}}.}

\textbf{Идея}: Играта се състои от плочки със стойности числата\\
$[1 \ldots n^2 - 1]$, разположени на $n \times n$ дъска, като една от позициите е празна.

Получаваме някаква конфигурация, която трябва да докараме до печеливша чрез последователно
побутване на плочки в свободната позиция.
\textit{Забележка:} Свободната позиция вътрешно се представя като числото 0.

\textbf{Печеливша} е само конфигурацията, в която елементите
са разположени в нарастващ ред от ляво надясно и от горе надолу, а празната позиция е долния десен ъгъл.

Към момента играта получава начално състояние и има възможността да се извежда дъската
след всеки ход. Вашата задача е да имплементирате функциите:
\begin{itemize}
    \item {
        \code{bool move(int tile)}, която премества дадената плочка в празното пространство,
        ако е възможно (и връща \code{true}). Ако такова преместване не е възможно,
        състоянието на дъската не се променя и се връща \code{false}.
    }
    \item {
        \code{bool won()}, която връща дали дъската е в печелившо състояние.
    }
\end{itemize}

След правилната имплементация на тези функции би трябвало играта да бъде
функционираща.
\end{document}
