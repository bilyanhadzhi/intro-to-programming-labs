\documentclass[12pt]{article}
\usepackage[utf8]{inputenc}
\usepackage[T1,T2A]{fontenc}
\usepackage[bulgarian]{babel}
\usepackage{inconsolata}
\usepackage{hyperref}
\usepackage{indentfirst}
\usepackage{multicol}

\hypersetup{
  colorlinks=true,
  linkcolor=blue,
  urlcolor=blue
}

\newcommand{\example}[2] {

    \smallskip
    \underline{Пример}:

    Вход:\hphantom{д} \code{#1}

    Изход: \code{#2}
}

\newcommand{\explanation}[1] {

    \underline{Пояснение}: #1
}

\newcommand{\outputline}[1] {\\\indent \indent #1}

\newcommand{\exercise}[2] {
    \textbf{Зад. #1.} #2
}

\newcommand{\latinttfamily}{\fontencoding{OT1}\ttfamily}
\DeclareTextFontCommand{\ltexttt}{\latinttfamily}
\newcommand{\code}[1]{\ltexttt{#1}}


\title{Увод в програмирането 2020/2021\\ Група 8, практикум №7}
\date{}

\begin{document}
\maketitle

\exercise{1}{Да се напише функция, която приема число и връща дали то е положително.}
\bigskip

\exercise{2}{Да се напише функция \code{max}, която приема две цели числа и връща стойността
на по-голямото от тях.}
\bigskip

\exercise{3}{
    Да се напишат две функции:
    \begin{itemize}
        \item {
            \code{input\_array}, която при подаден масив \code{int arr[SIZE]} и
            дължина \code{int n} въвежда елементите на \code{arr}.
        }
        \item{
            \code{print\_array}, която при подаден масив \code{int arr[SIZE]} и
            дължина \code{int n} извежда елементите на \code{arr}.
        }
    \end{itemize}
}
\bigskip

\exercise{4}{
    \begin{itemize}
        \item Да се напише функция, която проверява дали дадено цяло число \code{x} е просто.
        \item{
            Да се напише функция, която приема двата края на затворен интервал, и
            извежда всички прости числа в него.
        }

    \end{itemize}
}
\bigskip

\exercise{5}{
    \begin{itemize}
        \item Да се напише функция, която проверява дали дадено цяло число \code{x} е точна степен на дадено цяло число \code{n}
        \item Да се напише функция, която по дадени цели числа \code{k} и \code{n}
        $(1 \leq k, n \leq 100)$ и масив от цели числа \code{a}, определя дали в \code{a} има поне \code{k} числа, които са точни степени на \code{n}.
    \end{itemize}
}

\newpage

\exercise{6}{
    Да се напише функция, която приема масив от цели числа в интервала [1, 100]
    и неговата дължина и връща \textbf{късметлийско число} от него.
    Едно число наричаме \textit{късметлийско}, ако стойността му
    е равна на броя на срещанията му в масива.

    Ако няма късметлийски числа, върнатата стойност е -1. Ако има повече от едно,
    трябва да се върне най-голямото.
}
\begin{multicols}{2}
\example{[1, 1, 2, 2]}{2}
\example{[3, 3, 1, 2, 3]}{3}
\columnbreak
\example{[5, 6, 1, 1, 7]}{-1}
\example{[4, 4, 4, 4]}{4}
\end{multicols}
\bigskip

\exercise{7}{
    Да се реализира функция със следната сигнатура:

    \code{int steps(int[SIZE] x\_values, int[SIZE] y\_values, int n)},\\
    която приема два масива с \code{n} на брой съответно \code{x} и \code{y}
    координати на точки в равнината.

    Функцията трябва да връща за колко \textbf{стъпки} може да се обходят всички точки
    в последователността, в която са дадени, започвайки от първата.

    Една \textbf{стъпка} може да е едно хоризонтално/вертикално движение с дължина 1,
    или едно диагонално движение.
}

\end{document}
