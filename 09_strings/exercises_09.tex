\documentclass[12pt]{article}
\usepackage[utf8]{inputenc}
\usepackage[T1,T2A]{fontenc}
\usepackage[bulgarian]{babel}
\usepackage{inconsolata}
\usepackage{hyperref}
\usepackage{indentfirst}
\usepackage{multicol}

\hypersetup{
  colorlinks=true,
  linkcolor=blue,
  urlcolor=blue
}

\newcommand{\example}[2] {

    \smallskip
    \underline{Пример}:

    Вход:\hphantom{д} \code{#1}

    Изход: \code{#2}
}

\newcommand{\explanation}[1] {

    \underline{Пояснение}: #1
}

\newcommand{\outputline}[1] {\\\indent \indent #1}

\newcommand{\exercise}[2] {
    \textbf{Зад. #1.} #2
}

\newcommand{\latinttfamily}{\fontencoding{OT1}\ttfamily}
\DeclareTextFontCommand{\ltexttt}{\latinttfamily}
\newcommand{\code}[1]{\ltexttt{#1}}


\title{Увод в програмирането 2020/2021\\ Група 8, практикум №8}
\date{}

\begin{document}
\maketitle

\exercise{1}{
    Да се напише програма, която въвежда име и фамилия (не по-дълги от 128 символа)
    на човек и ги извежда във формат: \outputline{\code{<Last name>, <First name>}}
}
\example{Jason Statham}{Statham, Jason}
\bigskip

\exercise{2}{
    Да се напише функция:
    \outputline{\code{void reverse(char* str)}}, \\
    която обръща низ на място, т.е. без използване на допълнителни масиви.
}
\bigskip

\exercise{3}{
    Да се напише функция:

    \code{void caesar(char* text, int key, char* result)},

    която шифрира първия низ по \href{https://en.wikipedia.org/wiki/Caesar_cipher}{шифъра на Цезар}
    и записва резултата в \code{result}.
}
\bigskip

\exercise{4}{
    Да се напише функция, която приема низ и връща дължината на най-дългия му непразен
    подниз, съставен от само един уникален символ.
}
\begin{multicols}{2}
    \example{inbetw\underline{ee}n}{2}
    \columnbreak
    \example{\underline{p}}{1}
\end{multicols}
\newpage

\exercise{5}{
    Да се напише функция:

    \code{void vigenere(char* text, char* key, char* result)}.

    Функцията трябва да шифрира първия низ по
    \href{https://en.wikipedia.org/wiki/Vigen\%C3\%A8re_cipher}{шифъра на Виженер},
    използвайки втория низ като ключ, и да запише резултата в третия.
}
\bigskip

\exercise{6}{
    Да се напише функция, която изтрива всички срещания на даден низ \code{what}
    в даден низ \code{where}.

    Функцията да връща указател към позицията на първия изтрит подниз в \code{where} или
    \code{nullptr}, ако не е изтрит нито един подниз.
}
\example{
    \outputline{where: "aa bb aac ac"}
    \outputline{what:\hphantom{e} "aa"}
}{
    \outputline{where: "** bb **c ac"}
}
\bigskip

\exercise{7}{
    Да се напише програма, която въвежда низ с дължина до 256 символа и извежда следните
    статистики:
    \begin{itemize}
        \item{броят на думите;}
        \item{средната дължина на думите;}
        \item{най-дългата дума;}
    \end{itemize}
}
\bigskip

\end{document}
