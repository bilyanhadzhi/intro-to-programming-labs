\documentclass[12pt]{article}
\usepackage[utf8]{inputenc}
\usepackage[T1,T2A]{fontenc}
\usepackage[bulgarian]{babel}
\usepackage{inconsolata}
\usepackage{hyperref}

\hypersetup{
  colorlinks=true,
  linkcolor=blue,
  urlcolor=blue
}

\newcommand{\example}[2] {

    \underline{Пример}:

    Вход:\hphantom{д} \code{#1}

    Изход: \code{#2}
}

\newcommand{\explanation}[1] {

    \underline{Пояснение}: #1
}

\newcommand{\exercise}[2] {
    \textbf{Зад. #1.} #2
}

\newcommand{\latinttfamily}{\fontencoding{OT1}\ttfamily}
\DeclareTextFontCommand{\ltexttt}{\latinttfamily}
\newcommand{\code}[1]{\ltexttt{#1}}


\title{Увод в програмирането 2020/2021\\ Група 8, практикум №1}
\date{}

\begin{document}
\maketitle

\exercise{1}{Да се напише програма, която извежда \code{Hello, FMI!}\\ на стандартния изход.
}

\bigskip


\exercise{2}{При вход число от тип \code{double}, да се изведе
    цялата му част.
}

\bigskip

\exercise{3}{Да се напише програма, която извежда големината на променлива от всеки от следните
    типове:

    \code{char}, \code{short}, \code{int}, \code{long}, \code{float}, \code{double}.
}

\bigskip

\exercise{4}{При въведени цели числа \code{x} и \code{y}, да се изведе сбора им по следния начин:
}
\example{6 4}{6 + 4 = 10}

\bigskip

\exercise{5}{При вход положително цяло число \code{n}, да се изведе в конзолата
    колко секунди има в \code{n} дни.
}
\example{1}{86400}
\example{7}{604800}

\newpage
\exercise{6}{Напишете програма, която приема две цели числа и извежда в конзолата
    тяхното средно аритметично.
}
\example{8 59}{33.5}

\bigskip

\exercise{7}{При прочетени от стандартния вход три цели числа \code{a}, \code{b} и \code{c},
    да се изпише в конзолата дали за тях са изпълнени следните булеви условия:
    \begin{itemize}
        \item ${a^2 + b^2 = c^2}$;
        \item {
            дали сумата от последните цифри на числата \code{a} и \code{b}
            е равна на \code{c}.
        }
    \end{itemize}
}

\bigskip

\exercise{8}{Да се въведе символ в стандартния вход и да се изведе числената му стойност.
}
\bigskip

\exercise{9}{По вход малка латинска буква да се изведе главната латиснка буква,
    която ѝ съответства. Може да използвате
    \href{https://www.cs.cmu.edu/~pattis/15-1XX/common/handouts/ascii.html}{ASCII таблицата} за помощ.
}
\bigskip

\exercise{10}{Да се въведат две цели числа като променливи и да се разменият
техните стойности по два начина. (без да се използва \code{swap()})
}

\bigskip


\end{document}
