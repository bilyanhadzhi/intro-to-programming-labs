\documentclass[12pt]{article}
\usepackage[utf8]{inputenc}
\usepackage[T1,T2A]{fontenc}
\usepackage[bulgarian]{babel}
\usepackage{inconsolata}
\usepackage{hyperref}
\usepackage{indentfirst}

\hypersetup{
  colorlinks=true,
  linkcolor=blue,
  urlcolor=blue
}

\newcommand{\example}[2] {

    \smallskip
    \underline{Пример}:

    Вход:\hphantom{д} \code{#1}

    Изход: \code{#2}
}

\newcommand{\explanation}[1] {

    \underline{Пояснение}: #1
}

\newcommand{\outputline}[1] {\\\indent \indent #1}

\newcommand{\exercise}[2] {
    \textbf{Зад. #1.} #2
}

\newcommand{\latinttfamily}{\fontencoding{OT1}\ttfamily}
\DeclareTextFontCommand{\ltexttt}{\latinttfamily}
\newcommand{\code}[1]{\ltexttt{#1}}


\title{Увод в програмирането 2020/2021\\ Група 8, практикум №6}
\date{}

\begin{document}
\maketitle

\exercise{1}{
    Да се напише програма, която приема естествено число \code{n} и \code{n} на брой числа в масив, и
    извежда най-малкото и най-голямото от тях.
}
\example{
    \outputline{6}
    \outputline{7 3 -1 2 10 0}
}{
    \outputline{-1 10}
}
\bigskip

\exercise{2}{
    Да се напише програма, която приема естествено число \code{n} и \code{n} на брой цели числа, и
    извежда нейната най-дълга монотонно растяща подредица.
}
\example{
    \outputline{9}
    \outputline{-5 5 4 2 6 7 7 10 4}
}{
    \outputline{2 6 7 7 10}
}
\bigskip
\newpage

\exercise{3}{
    Да се напише програма, която приема естествено число \code{n} и \code{n} на брой цели числа.

    Да се изведат числата, които са по-големи от сбора на всички след тях.
    \textit{Забележка:} Сборът на нула числа е нула.
}
\example{
    \outputline{4}
    \outputline{5 1 2 1}
}{
    \outputline{5 2 1}
}
\explanation{5 > (1 + 2 + 1), 2 > 1, 1 > 0}
\bigskip

\exercise{4}{
    Да се напише програма, която извежда колко пъти се среща всяка цифра
    по подадено цяло число. Числото може да съдържа между 1 и 12 цифри.
}
\bigskip

\exercise{5}{
    \textbf{Колко ученици учат?}

    Да се напише програма, която приема естествено число \code{n}
    и \code{n} на брой двойки естествени числа.

    Всяка двойка е растяща и отговаря на период, в който даден ученик е учил.
    Първото число е моментът на започване, а второто - на приключване на ученето.
    Последно се въвежда точка във времето.

    Да се изведе колко на брой ученици учат в тази точка от времето.
}
\example{
    \outputline{3}
    \outputline{1 6}
    \outputline{2 4}
    \outputline{4 5}
    \outputline{4}
}{
    \outputline{2}
}
\explanation{
    \textit{Ученик 1} учи в 4.
    \textit{Ученик 2} завършва в 4, т.е. вече не учи.
    \textit{Ученик 3} започва в 4, т.е. вече учи.
}
\bigskip

\newpage
\exercise{6}{
    По въведен масив от \code{n} на брой естествени числа, да се
    преместят всички нули в края на масива, запазвайки наредбата
    на останалите елементи.
}
\example{
    \outputline{8}
    \outputline{5 2 0 3 7 0 1 0}
}{
    \outputline{5 2 3 7 1 0 0 0}
}
\bigskip

\exercise{7}{
    Да се напише програма, която по въведени \code{n} числа и число \code{k} проверява дали
    съществува подредица, сумата на числата в която е равна на \code{k}.
    Ако има, да се изведат индексите на първия и последния член от подредицата. В противен
    случай да се изведе подходящо съобщение.
}

\end{document}
