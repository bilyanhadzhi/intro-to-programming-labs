\documentclass[12pt]{article}
\usepackage[utf8]{inputenc}
\usepackage[T1,T2A]{fontenc}
\usepackage[bulgarian]{babel}
\usepackage{inconsolata}
\usepackage{hyperref}
\usepackage{indentfirst}
\usepackage{multicol}

\hypersetup{
  colorlinks=true,
  linkcolor=blue,
  urlcolor=blue
}

\newcommand{\example}[2] {

    \smallskip
    \underline{Пример}:

    Вход:\hphantom{д} \code{#1}

    Изход: \code{#2}
}

\newcommand{\explanation}[1] {

    \underline{Пояснение}: #1
}

\newcommand{\outputline}[1] {\\\indent \indent #1}

\newcommand{\exercise}[2] {
    \textbf{Зад. #1.} #2
}

\newcommand{\latinttfamily}{\fontencoding{OT1}\ttfamily}
\DeclareTextFontCommand{\ltexttt}{\latinttfamily}
\newcommand{\code}[1]{\ltexttt{#1}}


\title{Увод в програмирането 2020/2021\\ Група 8, практикум №11}
\date{}

\begin{document}
\maketitle

\exercise{1}{
    Да се напише рекурсивна функция, която смята сбора на две положителни числа.
}
\bigskip

\exercise{2}{
    Да се напише рекурсивна функция, която намира индекса на първата главна буква в низ.
    Ако няма такава, да се върне -1.
}
\bigskip

\exercise{3}{
    Да се напише рекурсивна функция, която връща дали низ е палиндром.
}
\bigskip

\exercise{4}{
    Да се напише рекурсивна функция, намира n-тото число на "Трибоначи".

    Редицата е дефинирана така:
    \begin{itemize}
        \item $T_0 = 0$
        \item $T_1 = 1$
        \item $T_2 = 1$
        \item $T_n = T_{n - 3} + T_{n - 2} + T_{n - 1},\ n \geq 3$
    \end{itemize}
}
\newpage

\exercise{5}{
    Да се напише функция, която приема масив от цели неотрицателни числа
    и начален индекс. Да се изведе дали чрез поредица от прескачания* в
    масива може да се стигне до индекс, на чиято позиция стои елемент със стойност 0.

    Ако се намираме на индекс \code{i}, то можем да прескочим към индекси:
    \begin{itemize}
        \item \code{i + arr[i]}
        \item \code{i - arr[i]}
    \end{itemize}
}
\example{
    \outputline{arr: [4, 2, 3, 0, 3, 1, 2], start: 5}
}{
    \outputline{1}
}
\explanation{Поредицата от индекси е: \textbf{5, 4, 1, 3}}
\bigskip

\exercise{6}{
    Да се напише функция, която търси решение на задачата за \textbf{разходката на коня}.
}

\end{document}
