\documentclass[12pt]{article}
\usepackage[utf8]{inputenc}
\usepackage[T1,T2A]{fontenc}
\usepackage[bulgarian]{babel}
\usepackage{inconsolata}
\usepackage{hyperref}

\hypersetup{
  colorlinks=true,
  linkcolor=blue,
  urlcolor=blue
}

\newcommand{\example}[2] {

    \smallskip
    \underline{Пример}:

    Вход:\hphantom{д} \code{#1}

    Изход: \code{#2}
}

\newcommand{\explanation}[1] {

    \underline{Пояснение}: #1
}

\newcommand{\outputline}[1] {\\\indent \indent #1}

\newcommand{\exercise}[2] {
    \textbf{Зад. #1.} #2
}

\newcommand{\latinttfamily}{\fontencoding{OT1}\ttfamily}
\DeclareTextFontCommand{\ltexttt}{\latinttfamily}
\newcommand{\code}[1]{\ltexttt{#1}}


\title{Увод в програмирането 2020/2021\\ Група 8, практикум №3}
\date{}

% if, switch, while, do while, bitwise, for

\begin{document}
\maketitle

\exercise{1}{
    Да се напише програма, която приема цяло число и извежда цифрите му наобратно.
}
\bigskip

\exercise{2}{
    Да се напише програма, която чете цели положителни числа от конзолата докато не се въведе \code{0}.
    Тогава от въведените числа да се изведат следните:
    \begin{itemize}
        \item най-голямото четно число;
        \item най-голямото нечетно число;
        \item сборът на всички числа;
        \item средноаритметичното на всички числа.
    \end{itemize}
}
\bigskip

\exercise{3}{
    Да се напише програма, която приема две цели числа и извежда
    като редица всички цели числа между тях, разделени със запетаи.

    Ако първото число е по-голямо, да се изведе намаляваща редица.
    Ако двете числа съвпадат, да се изведе само съвпадащата им стойност.

    За да е подредено, извеждайте не повече 10 числа на ред.
}
\example{25 25}{\outputline{25}}
\example{39 54}{
    \outputline{39, 40, 41, 42, 43, 44, 45, 46, 47, 48,}
    \outputline{49, 50, 51, 52, 53, 54}
}
\newpage

\exercise{4}{
    Да се напише програма, която приема цяло число \code{n} и след това \code{n} на брой цели числа.
    Да се изпише \code{yes}, ако съществуват две равни последователно въведени числа
    и \code{no} в противен случай.
}
\example{6 \outputline{1 7 4 4 3 15}}{
    yes
}
\bigskip

\exercise{5}{
    Да се напише програма, която приема цяло число \code{n} и
    извежда $n$-тото число на Фибоначи.

    Числата на Фибоначи са:
    \begin{itemize}
        \item fib(0) = 0
        \item fib(1) = 1
        \item fib($n$) = fib($n - 1$) + fib($n - 2$), $n > 1$
    \end{itemize}
}
\bigskip

\exercise{6}{
    Дадено е шестцифрено число, съставено само от цифрите 5 и 7.
    Разрешено ни е да променим само една негова цифра от 5 на 7 или обратно.
    Да се изведе най-голямото число, което можем да направим чрез това правило.
}
\example{55775}{75775}
\bigskip

\exercise{7}{
    Да се приеме цяло число \code{k} и да се изведе триъгълник от \code{k} реда в следния вид: (\textit{Floyd's triangle})
}
\example{4}{
    \outputline{1}
    \outputline{2 3}
    \outputline{4 5 6}
    \outputline{7 8 9 10}
}
\bigskip

\exercise{8}{
    Да се напише програма, която проверява дали едно число е просто.
}
\bigskip

\exercise{9}{
    Дадено ви е положително цяло число \code{n}.
    Разглеждайки числото в двоична бройна система, изведете:
    \begin{itemize}
        \item Броят битове със стойност 1 в числото
        \item Броят битове със стойност 0 в числото
        \item Дали числото е точна степен на двойката
        \item Дали числото е съставено от алтерниращи битове (т.е. всеки два съседни бита имат различни стойности)
    \end{itemize}
}
\example{4}{
    \outputline{Number of ones: 1}
    \outputline{Number of zeros: 2}
    \outputline{Is a power of 2: yes}
    \outputline{Has alternating bits: no}
}
\explanation{4 $\equiv$ $100_2$. 4 е степен на двойката и има две поредни нули.}

\example{10}{
    \outputline{Number of ones: 2}
    \outputline{Number of zeros: 2}
    \outputline{Is a power of 2: no}
    \outputline{Has alternating bits: yes}
}
\explanation{10 $\equiv 1010_2$, имаме алтерниращи битове.}

\end{document}
